\documentclass[letterpaper,twoside,onecolumn,report]{memoir}
\usepackage[utf8]{inputenc}
\usepackage[T1]{fontenc}
\usepackage{geometry}
\usepackage{graphicx}
\usepackage{xcolor}

\counterwithout{section}{chapter}
\setcounter{tocdepth}{2}
\setcounter{secnumdepth}{2}
\newcommand{\fixme}[1]{\textcolor{red}{\textit{#1}}}

\title{Rosetta Technote: protoDUNE Numbering Conventions}
\date{}

\usepackage[linktoc=all,pdftex,pdfpagelabels,bookmarks,hyperindex,hyperfigures]{hyperref}
\begin{document}
\maketitle

\begin{abstract}
  What it is and what it shall be.
\end{abstract}
\clearpage

\tableofcontents
\clearpage

\section{Introduction}

This document describes the conventions used to form
\textit{identifiers} for the \textit{elements} that make up the
single-phase protoDUNE (pD) experiment.  The main goal is to collect
existing conventions for \textit{physical} identifiers (see
section~\ref{sec:physlog}) and assert a common \textit{logical}
convention (again see section~\ref{sec:physlog}) which can be followed
by a large fraction of the experiment's \textit{domains}.

Experiment \textit{elements}, here, are atomic parts of the experiment
(eg, one sense wire segment) or are aggregations of elements (eg, one
contiguous sense wire conductor).  Different groups within the DUNE
collaboration develop their own conventions which cover their
particular \textit{convention domain}.  Domains may overlap between
groups and inconsistent conventions may evolve to cover these
overlaps.  

\subsection{Physical and Logical Identifiers}
\label{sec:physlog}

Identifiers, here, are concise numerical or alpha-numerical labels
given to elements of the experiment apparatus.  They are classified
into two distinct categories which are called here \textit{physical}
and \textit{logical} identifiers.  They are distinguished as:
\begin{description}
\item[physical identifiers] label each produced instance of some type
  of element.

\item[logical identifier] label a location or ``niche'' at which
  physical elements are placed in the experiment. 
\end{description}

\textit{Physical identifiers} are typically in the form of a serial
number or otherwise unique identifier assigned to each manufactured
part (or aggregation).  A physical identifier is uniquely determined
by the domain that produced them.  Eg, it is assumed that will there
never be more than one serial number assigned to a given physical
element.  Elements which are aggregates of other elements should
ideally be given physical identifiers that uniquely represent their
aggregation.  If an aggregate has an element replaced then the result
should be considered a uniquely new aggregate.  

\textit{Logical identifiers} indicate a ``niche'' or a ``role'' in the
detector which a physical element supplies.  A niche has a conceptual
location in the experiment.  This may be known in absolute terms (eg,
the most upstream APA on the right side of the beam) or in relative
terms (eg, wire segment 4 is next to wire segment 3).  An aggregate of
niches can itself be identified.  A conductor is composed of multiple
wire segments and an FEMB has multiple ASICs.  In most cases there
exists an \textit{array} of niches of common type (``wire segments in
a plane'' or ``channels in an ADC'').  In such cases it is desired to
produce a logical identifier which expresses an individual niche as
label identifying the array and an offset into that array (eg, ``U
wire 42'').  Ideally this offset is 1-based so that 0 may be reserved
to indicate an error.  However, for some this 1-base can lead to
confusion in software and is already in conflict with some established
conventions and so 0-based array offsets are used.

\subsection{Associations}

While a physical identifier is (ideally) uniquely determined by the
domain that fabricates it, a logical identifier may differ among the
domains that must refer to it.  Even within one domain, identifiers
may differ over time (eg, different versions of detector geometry used
in simulations).  Giving explicit identifiers being assigned to
physical or logical aggregates, each set of identifiers can be
organized into a hierarchy.  As physical elements are installed into
logical niches a mapping between a physical hierarchy and any given
logical hierarchy is formed.  This mapping must be traversed in both
directions.  And, this mapping must evolve during installation and
potentially later if parts fail or are otherwise replaced.

Besides aggregation (``this WIB is in this WIEC''), logical
identifiers may be associated through \textit{connectivity}.
Connectivity can be between elements of the same type (``this APA is
adjacent to this other APA'') and elements of different types (``this
wire conductor is connected to this FE ASIC channel'').

Both physical and logical identifiers have associated information.
Physical identifiers can have associated performance or QA/QC
parameters.  Logical identifiers can have associated quantities giving
the location and extent taken by the physical part.

These associated parameters may change over time and between domains.
Maintaining these associations is important but is not considered part
of the scope for the identifier conventions themselves.

\section{Domain Conventions}

As described above, this note assumes one can segment the problem into
\textit{convention domains}. 
This document is restricted to a subset of all domain conventions
that are relevant to the analysis of the experiment data and for
communicating with domain experts any issues that may be found in the
course of analyzing the experiment data.  Experiment data, for the
purposes of this note, consists of that directly acquired by the
detector DAQ, the simulation of this data as well as any acquired
through quality assurance (QA) and control (QC) processes (eg cold
electronics testing).

Ideally, a domain is responsible for assigning a globally unique
\textit{physical} identifier to each part (including an aggregation)
that it produces.  However, it is possible for parts to be passed
between domains for further aggregation and new physical identifiers
invented and assigned.  This becomes a bookkeeping challenge for the
domains involved if or when there is a need to communicate issues back
to the originating domain.

Any domain may develop its own convention for \textit{logical}
identifiers although it is desirable, and the goal of this note, to
minimize this variability.

The remaining part of this section collects the conventions relevant
to the considered domains.  \fixme{What needs changing?}
In section~\ref{sec:onering} a common
logical convention is asserted.  In section~\ref{sec:impl} issues of
implementation in support of this convention are discussed.

\subsection{Engineering}
\label{sec:eng}



\subsection{Quality Assurance and Control}
\label{sec:qaqc}

\subsection{Construction and Installation}
\label{sec:inst}

\subsection{Slow Control}
\label{sec:slow}

\subsection{Electronics}
\label{sec:elec}

\subsection{DAQ}
\label{sec:daq}

Includes TPC, PDS, TI, CR fragments and oob BI.  Starts with BR, ends
with EB and writing out a raw file.

\subsection{Offline}
\label{sec:offline}

Simulation and Geometry.  LArSoft data products.  Offline databases?

\subsubsection{LArSoft}

\begin{itemize}
\item protoDUNE-SP Geometry coordinate system, CPA/APA/TPC/wires \href{https://cdcvs.fnal.gov/redmine/projects/dunetpc/wiki/DUNE_Geometries#ProtoDUNE-SP-Geometry}{(link)}
\end{itemize}

\section{Common Convention}
\label{sec:onering}

Write this section once enough from the domains are known.  But, it
will probably draw heavily from what LArSoft does now.

Include table of as-designed multiplicities (asic per femb, channel per asic, etc).

\section{Implementation Issues}
\label{sec:impl}

What work needs to be done.

\end{document}
