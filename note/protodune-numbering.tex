\documentclass[letterpaper,twoside,onecolumn,article]{memoir}
\usepackage[utf8]{inputenc}
\usepackage[T1]{fontenc}
\usepackage{geometry}
\usepackage{graphicx}
\usepackage{xcolor}
\usepackage{xspace}

\counterwithout{section}{chapter}
\setcounter{tocdepth}{2}
\setcounter{secnumdepth}{2}
\newcommand{\fixme}[1]{\textcolor{red}{\textit{#1}}}
\newcommand{\pD}{pD/SP\xspace}

\title{Rosetta Technote: protoDUNE Numbering Conventions}
\date{}

\usepackage[linktoc=all,pdftex,pdfpagelabels,bookmarks,hyperindex,hyperfigures]{hyperref}
\begin{document}
\maketitle

\begin{abstract}
  What it is and what it shall be.
\end{abstract}
\clearpage

\tableofcontents
\clearpage



\section{Todo}
\begin{itemize}
\item \fixme{discuss more multi-dimensional identifiers vs. flat, global numbering.}
\item \fixme{list and name the different conventions presented.}
\item \fixme{discuss strategy of reserving the number zero.}
\item \fixme{factor out narrative from convention definition.}
\item \fixme{factor out coordinate systems into one section.} 
\end{itemize}

More ``fixmes'' are scattered throughout the text.

\section{Introduction}

This document describes the conventions used to form, typically
numerical, \textit{identifiers} for the \textit{elements} that make up
the single-phase protoDUNE (\pD) experiment.  Specifically, identifiers
for the location or ``niche'' of elements in the in-situ and
as-designed detector are developed.  These locations are ideal (eg,
uniform wire pitch) and sometimes logical (eg, $4^{th}$ ADC on a
FEMB).  Serial numbering conventions of actual instances of elements
is not considered.

It is assumed that the convention described hee will be used by
analyzers of data in an offline software context.  As such, it is
expected that the raw data files from the DAQ and subsequent derived
files will follow this convention.  As such, the convention is
developed to adhere, where applicable, to that used by the producers
of the raw data.  A secondary goal of this document is to relate the
convention to any others used by other DUNE groups in order to
facilitate communication between them and the offline analysis
efforts.

The remaining sections of this document are cover the major subsystems
of \pD: the time projection chamber (TPC), the photon detection system
(PDS) and the trigger information (TI) system including beam and
cosmic particle sources.  Finally, the labeling applied by the DAQ
as it writes the raw data to file is described.

\section{\pD Overview and Coordinates}

\fixme{Add table summarizing quantities.}

\fixme{Add origin.}

This section gives an overview of \pD in order to define the terms
used in this document and to illustrate the coordinate system used to
describe the physical locations.

Figure~\ref{fig:protodune} shows a high level illustration of the
major elements of \pD and the \textit{external coordinate system}.
The $Z_{ext}$ direction runs the length of \pD and is in the direction
of the major component of the beam direction.  $Y_{ext}$ is upward
against the force of gravity and $X_{ext}$ follows from the
right-handed cross product of $X_{ext} \times Y_{ext} = Z_{ext}$.

\begin{figure}[h]
  \centering
  \fixme{Find or make figure}
  \caption{Major elements of \pD and the convention for the \textit{external coordinate system}.}
  \label{fig:protodune}
\end{figure}


\fixme{Add comparison to LArSoft's current coordinate system.}


Todo:
\begin{itemize}
\item wires and conductors
\item G, U, V and W wire plane designations.
\item PDS.
\item Beam.
\item CR.
\end{itemize}

\section{TPC Numbering Convention}

This section describes the numbering convention covering the TPC data.
It is described in terms of two different but related views: wire and
channel where one can consider the flow of information about activity
in the detector to flow into wires and out of channels.  In that same
order, the wire-related elements are:

\begin{description}
\item[wire segment] (or as a noun simply \textbf{wire}), is a straight
  conductor that runs across one face of one APA with at a given angle
  measured w.r.t. the vertical ($Y_{ext}$).  The multiplicity of wires
  is described below.
\item[wire conductor] or simply \textbf{conductor}, one, two or three
  wires may be connected through a trace on an edge board strip to
  form a longer conductor which then effectively causes a wrapping
  around the APA.
\item[wire segment] or simply \textbf{segment}, counts the wires in a
  conductor starting with segment 0 as the wire attached to the top of
  the APA.
\item[wire plane] collects all wires on one side of an APA which run
  parallel to each other.
\item[wire face] each side of an APA is labeled ``inner'' or ``outer''
  to reflect whether it faces inward to or outward from the general
  center of the detector.  Each of two wire faces has a stack of three
  parallel wire planes.  Here, these are labeled, in the order in
  which the drifting electrons encounter them, ``U'', ``V'' and ``W''
  (engineering drawings typically label this latter plane ``X'' and
  some offline software labels it ``Y'').
\end{description}

The wire conductor is attached to the top of the APA and starting from
that point the channel-oriented elements are:

\begin{description}
\item[channel plane] is a logical plane taking the same labels as a
  wire plane however is defined by the conductor and thus subject to
  wrapping.  Each of the two U (and V) channel planes cover half of
  each APA face.  Each of the two X channel planes cover one APA face
  (no wrapping).
\item[channel face] is also labeled ``inner'' or ``outer'' and
  collects the three channel planes of conductors which terminate on a
  given side of an APA.
\item[channel board] each channel face is divided into a line of ten
  boards (actually assemblies of several types of boards).  Into each
  board goes 40, 40 and 48 conductors from the U, V and W channel
  planes of the associated channel face.
\item[channel group] each channel board divides its 128 conductors
  into eight groups of 16 channels.  Strictly, there is also a
  subgroup division due to two internal connectors being used but this
  will be neglected here.
\item[channel] finally a each of the 16 conductors in a channel group
  is fed into one of the 16 channels on the front-end (FE) ASIC for
  amplification and shaping and from there to one of one of the 16
  channels of the ADC ASIC.
\item[...] \fixme{continue through COLDATA, warm FPGA, DAQ}
\end{description}

The remaining subsection will describe in detail the numbering
convention related to these parts.

\subsection{Wire and Drift Coordinate Systems}

In addition to the external coordinate system described above and
shown in figure~\ref{fig:protodune} there are two more types of useful
coordinate systems which are illustrated in
figure~\ref{fig:coordsystems}.

\begin{figure}[h]
  \centering
  \fixme{make this figure}
  \caption{Illustration of \textit{drift cell coordinate system} and one of three \textit{wire plane coordinate systems}}
  \label{fig:coordsystems}
\end{figure}

The first type is tied to the LAr drift cell ($dc$) and its APA face.
It exploits the translational+rotational symmetry of \pD and the fact
that, by construction, each LAr drift cell operates essentially
independent from others in terms of the physics of electron drift and
field response.  Using the $dc$ coordinate system allows large
analysis and simulation codes to adopt a divide and conquer strategy
and identical code to operate on six otherwise identical
sub-detectors.

The second type of coordinate system that is important, especially for
simulation is one tied to the geometry of each wire plane.  The wire
plane coordinate systems are the basis for calculating detector field
response due to distributions of electrons drifting near a given wire
plane.  This allows a dimensional reduction of a hard 3D problem into
three easier 2D problems.  The rest of this section describes these
two types of coordinate systems, how they interrelate and how they
relate to the external coordinate system.

As described above, a wire (segment) is a straight run of a conductor
across one side of an APA.  The direction of a wire plane
$i \in [\mbox{U}, \mbox{V}, \mbox{W}]$ and its wires, is given by a
unit 3-vector $\hat{w}_i$.  The sign of this $\hat{w}_i$ is chosen to
point in the direction that signals travel on the wires toward the
electronics.  That is, it has a positive component along the $Y_{ext}$
axis.

All neighboring wires in a wire plane $i$ are nominally separated by a
fixed distance as measured perpendicularly to their wire direction.
This distance is called the \textit{wire pitch} and the
pitch-direction, $\hat{p}_i$ is also represented by a unit 3-vector
chosen to point generally toward the left if one faces a plane looking
in the direction of electron drift.

Along the pitch direction, the wires represent periodic values.  A
choice of \textit{pitch phase} shifts these values by some amount
along the pitch direction.  \fixme{What is this phase for each plane
  and how was it chosen?  Given that, how best to define it? (relative
  to origin?  relative to frame corner?)}

With this convention for wire and pitch directions, their cross
product $\hat{w}_i \times \hat{p}_i$, points counter to the nominal
electron drift direction.  This \textit{anti-drift} direction is
identified with that of the $X$-axis of a \textit{drift cell
  coordinate system}, $X_{dc}$.  The $Y_{dc}$ axis of the drift cell
coordinate system is parallel to the $Y_{ext}$-axis and to
$\hat{w}_{\mbox{W}}$.  Finally, the $Z_{dc}$ axis points according
their cross product which is also parallel to $\hat{p}_{\mbox{W}}$.

The origin of the drift cell and the three wire plane coordinate
systems are chosen to coincide with each other and their counterparts
that are associate with the opposite side of the APA.  This origin is
centered on the APA.  Longitudinally, is it half way between the the
two physical faces of any given channel plane.  Transversely, it is at
the geometric mean of the APA \textit{active area}.

Table~\ref{tab:wire-drift-coord-params} summarizes the geometrical
quantities relevant to these coordinate systems.


\begin{table}[h]
  \centering
  \begin{tabularx}{1.0\linewidth}[h]{|r|l|}
    angles & u,v,w \\
    pitches & u,v,w \\
    phases & u,v,w \\
    wp axis directions & in dc coord \\
    active width & \\
    active height & \\
  \end{tabularx}
  \caption{Parameters relevant to drift cell and wire plane coordinate systems.}
  \label{tab:wire-drift-coord-params}
\end{table}

\fixme{Describe how LArSoft has no coordinate systems like these (right?)}

\subsection{Wire Numbering}

The wires in a plane represent an array of objects.  Many algorithms
require fast indexing into this array, especially when coupled with
the associated wire plane coordinate system described above.  The
origin of this array is chosen to be the wire with the most negative
pitch as measured along the associated pitch direction.  This puts
``wire 0'' of an plane at the right-most side or corner as determined
by a viewer looking in the electron drift direction.

\fixme{As far as I know, LArSoft doesn't have such numbering.  Yes?}

\subsection{Channel Boards}

As described above, a channel board accepts 40 U, 40 V and 48 W wires.
Not mentioned yet but it also accepts 48 grid or ``G'' wires which are
up-drift of the U plane and parallel to the W wires.  A channel board
is actually a sequence or assembly of different boards:

\begin{description}
\item[wire board stack] one wire board for each of the four types of wires.
\item[CR board] holds capacitors and resistors to provide isolation and wire bias voltage.
\item[adapter board] adapts between the CR board and the cold electronics box.
\item[front-end board] the front-end mother board in the cold
  electronics box houses the eight FE amplifier and eight ADC ASICs
  and the cold FPGA.\fixme{I need to add COLDATA, right?}
\end{description}

The electronics group has a convention~\cite{docdb2060} that covers
just the context of one channel board.  Facing an APA and looking at
an installed channel box this convention counts wires from left to
right.  This convention makes a triple mapping between pins in the two
96-pin Samtech connectors on the adapter board, a wire plane
and number and an ASIC (FE or ADC) and its channel as summarized in
table~\ref{tab:board-wire-chan}.

\begin{table}[h]
  \centering
  \begin{tabular}[h]{|r|l|}
    \hline
    Element & values \\
    \hline
    Samtech connectors & J1, J2 \\
    Channel planes & U, V, and W (called ``X'') \\
    Board-local wire numbers & 1-40 (U and V), 1-48 (W) \\
    ASICs (called ``group'') & 1-8 \\
    ASIC-local channel number & 0-15 \\
    \hline
  \end{tabular}
  \caption{Summary of board-level wire/channel mapping from reference~\cite{docdb2060}.}
  \label{tab:board-wire-chan}
\end{table}

The electronics group convention does not cover how the boards
themselves are numbered as they go along the top of both APA
faces. \fixme{Right?  Who does?}.  Here, facing the APA, they line of
boards on the nearest face are numbered from 1 -- 10 going left to
right.  This is the same direction as the board-local wire numbers
increase.  This allows an flat numbering convention for each channel
plane to be defined as

\begin{equation}
  \label{eq:wire-attachment-numbers}
  w_{i} = (B-1) * N_{i} + w_{B,i}
\end{equation}

Here, $w_{i}$ is a \textit{wire attachment number} for the given APA
face and channel plane $i \in [\mbox{U}, \mbox{V}, \mbox{W}]$, $B$ is
a board number, $N_{i} = (40, 40, 48)$ is the number of board-local
wires for a given plane and $w_{B,i}$ is the board-local wire number
for board $B$.  The ranges for these wire attachment numbers are then
$w_{\mbox{V}}, w_{\mbox{U}} \in [1, 400]$ and
$w_{\mbox{W}} \in [1, 480]$.  This convention applies to both faces of
all APAs.

The numbering of ASICs can be similarly flattened across an APA
channel face by:

\begin{equation}
  \label{eq:adc-face-numbers}
  ADC = (B-1) * 8 + ADC_{B}
\end{equation}

With $ADC_{B}$ being the board-local ADC number and $ADC \in [1-80]$.
However, be cautious that there is a ``nonlinear'' mapping between an
$ADC$ numbers and the wires its services.  A diagonal (in longitudinal
drift time and wire-transverse space) track will alternately ``hit''
pairs of ASICs in order: (2, 4), (3, 1), (5, 7) and finally (8, 6).



\section{PDS}

\begin{enumerate}
\item paddles
\item detectors
\item cable terminations
\item channels
\item APAs
\item \pD
\end{enumerate}

\section{Trigger}

\begin{itemize}
\item Run definitions and identifiers
\end{itemize}

\subsection{Beam}

\begin{itemize}
\item beam holes and angles
\item run plan info (particle content, momenta)
\end{itemize}

\subsection{Cosmic Ray}

\begin{itemize}
\item paddle location and numbering
\end{itemize}

\subsection{DAQ}

\begin{itemize}
\item labels related to artDAQ nodes
\item file naming
\item payload schema?
\end{itemize}

\end{document}
